\section{25.10.2024 лекция 8}
\subsection{К}

\begin{minted}{asm}
; prim3.asm - обращение к полям структур: цикл в цикле для работы с 2-мя структурами
Start: mov AH, 9
        mov DX, offset message
        int 21h
    lea BX, st1     ;
    mov CX,2
m2: push CX
    mov SI, 0
    mov CX, 3
m1: push CX 
        lea DX, [BX][SI]
            int 21h
            add SI, 9
            pop CX
            loop m1
add BX, type tst

pop CX 
    loop m2
    ret
messsage DB "hello", 0dh, 0ah, "$"
    lst struc  
            s DB ?
            f DB ?
            i DB ?
    tst ends
            st1 tst <"student$", "$",>

\end{minted}

\subsection{Записи в Ассемблере}
Запись в Ассемблере, также как и структура состоит из различнх данных различной длины, но запись "--- упакованные данные, занимающие не полные ячейки памяти (байты, слова), а их части. Поля записи представляют собой последовательности битов(разрядов), прижатые друг к другу, между полями не может быть пробелов, размер полей мождет быть различным, но в сумме их размер не должен превышать 16 разрядов. Т.е. сумма размеров полей "--- это размер записи, а размер записи может быть 8 или 16 рарзрядов. Если сумма разрядов полей меньше 8 или 16, то поля должны быть прижаты к правой границе поля, а лишние левые разряды равны 0.  К записи не имею отношения, не влияют на неё и не используются.

Чтобы использовать запись, нужно вначале описать её тип, а затем описать переменную такого типа. Описание типа записи может располагаться в любом месте, но до описания переменных. \textbf{ВАЖНО! Поля в записи имеют собственные имена, располагаются в памяти, как при описании, слева направо. Но обращаться к полям записи, как к полям структуры напрямую нельзя.}

Директива описания типа записи имеет вид:
\begin{center}
    <имя типа записи> record <поле> {, <поле>}
    <поле>::=<имя поля>:<размер>[=<выражение>]
\end{center}
Здесь <размер> и <выражение> "--- константные выражения.

<размер> определяет размер поля в битах, <выражажение> определяет значение поля по умолчанию. Знак ? не допускается.

Например:
==

Год, записанный двумя последними цифрами
Имена полей, также как и в структурах, должны быть уникальными в рамках всей программы, в описании они перечисляются слева направо. <Выражение> может отсутствовать, если оно есть, то его значение должно умещаться в отведенный ему размер в битах. Если для некоторого поля выражажение отсутствует, то его значение по умолчанию равно нулю, неопределенных полей не может быть.

Определение директивой record имя типа(Trec, TData) использутется далее как директива для описания переменных-записей такого типа.
\begin{center}
    имя записи имя типа записи <начальные значения>,
\end{center}
угловые скобки здесь не метасимволы, а символы языка, внутри которых через запятую указываются начальные значения полей.

Начальными значениями могут быть:

В отличии от структуры, знак ? определяет нулевой начальное значение, а <<пусто>>, как и в структуре, определяет начальное значение равнм значению по умолчанию. Например:

Также, как и для структур:
\begin{center}
    1
\end{center}
Одной директивой можно описать массив записей, используя несколько параметров в поле операндрв или конструкцию повторения, например, 
\begin{center}
    1
\end{center}
Описали 100 записей с начальными значениями, равными принятыми по умолчанию.

Со всей записью в целом можно работать как обычно с байтами или со словами, т.е. можно реализовать присваивание Rec1=Rec2:
\begin{minted}{asm}
    mov AL, Rec2
    mov Rec1, AL
\end{minted}
Для работы с отлельными полями записи существует специальные операторы \textbf{1}

оператор mask имеет вид:
\begin{center}
    Mask <имя поля записи>
    Mask <имя записи или имя типа записи>
\end{center}
Значением этого оператора является <<маска>> "--- это байт или слово, в зависиморсти от размера записи, соедржащее единицы в тех разрядах, которые принадлежат полю или всей записи, указанных в качестве операнда, и нули в остальынх, не

Пример. Выявить родившихся 1-го числа, для этого придется выделять поле D и сравнивать его значение с 1-ей
\begin{minted}{asm}
    m1:
        mov AX, Dat1
        and AX, mask D 
        cmp AX, 1
        je yes
    no:
        

        jmp 
\end{minted}

Обьединение (union) является тином данных, состоящим из нескольких переменных, которые хранятся, начиная с одного и того же адреса памяти (перекрывая друт друuа). Этот тип данных есть во многих языках программирования, например, в языке С. Пример на Ассемблере: 
\begin{minted}{asm}
    pdate record d:5, m:4, y:7=4
    udate strus
            d db ?; локальное имя!
            m db ?
            y dw 4
    udate ends 
    date union
            Dp pdate <>;2 байта
            Du udate <>;4 байта
            7, dw 3 dup (?); 6 байт
    date ends

    MyDate date <>
    mov ax,MyDate.Dp;2 байта
    and ax,Mask d; d - глобальное имя из записи pdate
    mov al, MyDate.Du.d
    mov ax, MyDate.Z; ax := Z[0] 
    mov ax, sizeof MyDate; sizeof MyDate=6
\end{minted}
В данном случае имена полей в объединении, как и имена полей в структурах, локализованы внутри обьединения. Объединения используются в основном для экономии места в памяти при хранении данных.

На рисунке показано расположение полей объединения с именем MyDate в памяти.

(фото 1)

Наложение полей объединения друг на друга

\subsection{Работа с подпрограммами}
Программа, оформленная как процедура, к которой обращение происходит из ОС, заканчивается командой возврата ret;

Подпрограмма, как вспомогательный алгоритм, к которому возможно многократное обращение с помощью команды саll, тоже оформляется как процедура с помощью директив proc и еndр. Структуру процедуры можно оформить так:
\begin{minted}{asm}
    <имя процедуры> рroc <параметр>

                    <тело процедуры>

                ret

    <имя процедуры> endp
\end{minted}
В Ассемблере один тип подпрограмм "--- процедура. Размещать ее можно в любом месте программы, но так, чтобы управление на нее не попадало случайно, а только по команде саll. Поэтому описание ПП принято располагать в конце программного сегмента (после последней исполняемой команды), или вначале его "--- перед первой псполняемой командой.

Графическое представление(фото)

\subsubsection{Замечания}
\begin{enumerate}
    \item После имени в директивах proc и endp двоеточие не ставится, но имя считается меткой, адресом первой исполняемой команды процедуры.
    \item Метки, описанные в ПП, не локализуются в ней, поэтому они должны быть уникальными в рамках всей программы.
    \item Параметр в директиве начала процедуры один "--- FAR или NEAR
\end{enumerate}

Основная проблема при работе с ПП в Ассемблере "--- это передача параметров и возврат результатов в вызывающую программу.

Существуют различные способы передачи параметров: 
\begin{enumerate}
    \item по значению
    \item по ссылке 
    \item по возвращаемому значению
    \item по результату
    \item отложенным вычислением
\end{enumerate}
Параметры можно передавать: 
\begin{enumerate}
    \item через регистры
    \item в глобальных переменных
    \item через стек \textbf{(наиболее универсальный, используется в языках высоко уровня)}
    \item в потоке кода
    \item в блоке параметро
\end{enumerate}
Передача параметров через регистры "--- нанболее простой способ. Вызывающая программа записывает в некоторые регистры фактические параметры\dots

Процедура получает адрес начала этого блока при помощи любого из рассмотренных методов: в регистре, в переменной, в стеке, в коде пли даже в другом блоке параметров. Примеры использования этого способа многие функции QC и BIOS, например, поиск файла, использующий блок параметров DIA, или загрузка и исполнение программы, использующая блок параметров EPB

\textbf{Передача параметров по значению и по ссылке.}

При передаче параметров по значению процедуре передается значение фактического параметра, оно копируется в І , и І использует копию, поэтому изменение, модификация параметра оказывается невозможным. Этот механизм используется для передачи параметров небольшого размера.

Например, нужно вычислить с = тах (a, b) + max (7, a-1). Здесь все числа знаковые, размером в слово. Используем передачу параметров через регистры Процедура получает параметры через регистры АХ ц ВХ, результат возвращает в регистре AX.

\subsection{Передача параметров по ссылке}
Оформим как процедуру вычисление х = x div 16

Процедура имеет один параметр-переменную х, которой в теле процедуры присваивается новое значение. Т.с результат записывается в некоторую ячейку памяти. И чтобы обратиться к процедуре с различными параметрами, например, а ц b, ей нужно передавать адреса памяти, где. хранятся значения переменных з и в. Передавать адреса можно любым способом, в том числе и через регистры. Можно использовать различные регистры, но чаще используются ВХ, BP, SI, DI. Пусть адрес параметра передается через регистр ВХ, тогда фрагмент программы:
\begin{minted}{asm}
    ;основная программа
    =============
    lea BX, a 
    call Proc_dv 
    lea BX, b 
    call Proc_dv 
    =============
\end{minted}
\begin{center}
    \textbf{Процедура:}
\end{center}
\begin{minted}{asm}
    push SX mov CL. 4
    shr word ptr [BX], CL; x = x div 16
    pop CX 
    ret
\end{minted}