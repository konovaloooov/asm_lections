\section{13.09.2024 лекция 2}
Специальным образом реализуется и используется сегмент стека


фото

Стек используется для временного хранения данных, для организации работы с подпрограммами

Для того, чтобы стек можно было использовать для хранения и фактических и локальных параметров, после перердачи фактических параметров значение указателя на вершину стека можно сохранить в регистре ВР и тогда к глобальным параметрам можно обратиться, используя инструкцию ВР + к, а к локальным - 

Регистр флагов. Регистр FLAGS или EFLAGS определяет состояние процессора и программы в каждый текущий момент времени.
В реальном и защищенном режиме
CF "--- Перенос
PF "--- четность
AF "--- полуперенос
ZA "--- флаг нуля
SF "--- флаг знака
TF "--- флаг трассировки
IF "--- флаг прерывания
DF "--- флаг направления
OF "--- флаг переполнения

Только в защищенном режиме:
\begin{itemize} 
    \item AC "--- флаг выравнивания операндов
    \item VM "--- флаг виртуальных машин
    \item RF "--- флаг маскировки прерывания
    \item NT "--- флаг вложенной задачи
    \item IOPL "--- флаг
\end{itemize}
VIF "---, VIP, ID

\subsection{Регистр флагов}
CF устанавливается в единицу, если в рез-те выполнения операции(например, в рез-те сложения перенос из старшего разряда, а при вычитании - заём)
PF = 1, если в младшем байте результата содержится четное кол-во единиц
AF = 1, если в рез-те выполнения команды сложения(вычитания) осуществляется перенос(заём) из 3 разряда байта в 4 (из 4 в 3)
ZF = 1, если все разряды результат окажутся равны 0
SF всегда равен знаковому разряду результата
TF = 1, прерывает работу процессора после каждой выполненной команды(пошаговая откладка программы)
IF "--- если сбросить этот флга в 0, то процессор перестанет обрабатывать прерывания от внешних устройств. Обычно его сбрасывают на короткое время для выполнения критических участков программы
DF определяет направление обработки строк данных, если DF = 0 "--- обработка строк идет в сторону увеличения адресов 1 "--- в сторону уменьшения(от старших к младшим). Автоматическое увеличение или уменьшение содержимого регистра указателей индексов SI и DI.
OF = 1, если результат команды превышает саксимально допустимый для данной разрядной сетки
IOPL = 1, если уровень привелегии текущей программы меньше значенгия этого флажка, то выполнение команды ввод/вывод для этой программы запрещен
NT - определяет режим работы вложенных задач
Флаги в защищенном режиме мы использовать не будем, рассказала вскользь.
\subsection{Регистры 64-х битного процессора}
16 целочисленных 64 битных регистра общего назначения(RAX, RBX, RCX, RDX, RBP, RSI, RDI, RSP, R8 "--- R15)
8 80 битных

\subsection{Оперативная память}
Состоит из байтов, каждый байт состоит из 8 информационных битов
0 - 3 цифровая часть байт
4 - 7 зонная часть байта
32 - х разрядный процессор может работать с ОП до 4 ГБайт и, следовательно, адреса байтов изменяются от 0 до $2^{32}$ ст - 1
$00000000_{16} - FFFFFFFF_{16}$

Байты памяти могут объединяться в поля фиксированной и переменной длины

Фиксированная длина - слово(2 байта), двойное слово(4 байта). Поля переменной длины могут содержать 

Адресом поля является адрес младшего входящего в поле байта. Адрес поля может быть любым.
ОП может использоваться как непрерывная последовательность байтов, так и сегментированная.

физический адрес(ФА) записывается как: 
<сегмент>: <смещение>, т.е. он может быть получен по формуле ФА = АС + ИА, где АС - адрес сегмента, ИА - исполняемый адрес, т.е. ИА - смещение 

В защищенном режиме программа может определить до 16383 сегментов размером до 4 ГБайт, и таким образом может работать с 64 терабайтами виртуальной памяти.

Для реального режима АС определяется сегментным регистром и для получения двадцатиразрядного двоичного адреса байта необходимо к содержанному сегментного регистра, смещенного на 4 разряда влево, прибавить шестнадцатиразрлное спещение ИА. Например, адрес следующей выполняемой команды:

ФА = (CS) + (IP)

(CS) = $7A15_{16}$ = 0111 1010 0001 0101 0000

(IP) = $C7D9_{16}$ =      1100 0111 1101 1001

ФА = $86929_{16}$ = 1000 0110 1001 0010 1001

\subsection{Оперативная память}
Процессор ix86 вместе с сопроцессором могут обрабатывать большой набор различных типов данных: целый числа без знака, целые числа со знаком, действительные числа с плавающей точкой, двоично-десятичные числа, символы, строки, указатели.

Целые числа без знака могут занимать байт, слово, двойное слово и принимать значени из диапазонов:
0 - 255, 0 - 65535, 0 - 4294967295

Дополнительный код положительного числа равен самому числу.
Дополнительный код отрицательного числа в любой системе счисления может быть получен по формуле:

$Х = 10^{n} - [X]$, где 




Числа с плавающей точкой могут занимать 32 бит, 64 бит, или 80 бит, и называется короткое вещественное, длиное вещественное, рабочее вещественное. Формат числа с плавающей точкой состоит из трёх полей:

<знак числа>, <машинный порядок>, <мантисса>.
\begin{itemize}
    \item Короткое вещественное $1 + 8 + 23 - 10 ^{+-32} - + 10 ^{+-32}$
    \item Длинное вещественное $1 + 11 + 52 - 10 ^{+-308} - + 10 ^{+-308}$
    \item рабочее вещестыенное $1 + 15 + 64 - 10 ^{+-4932} - + 10 ^{+-4932}$
\end{itemize}

Машинный порядок (Пм) включает в себя неявным образом знак порядка и связан с истинным порядом (Пи) формулой:

Пм = Пи + 127 10 (1023 10 16383 10)

Предполагается, что мантисса нормализована в старший и старший единичный разряд мантиссы не помещается в разрядную сетку.

Двоично - десятичные чисоа могут обрабатываться 8-ми разрядные в упакованном и неупакованном формате, и сопроцессором могут обрабатываться 80 разрядные данные в упакованном формате. Упакованный формат предполагает хранение двух цифр в байте, а неупакованнй - хранит одну цифру в цифровой части байта.

\subsection{Форматы данных}
Символьные данные - символы в виде ASCII. Для любого символа отводится один байт.
Строковые данные - это последовательности байт, слов или двойных слов.
Указатели - существуют два типа указателей: 
длинный, занимающий 48 бит - селектор(16) + смещение(32)
и короткий указатель, занимающий 32 байта - только смещение
\subsection{Форматы команд}
Команда - это цифровой двоичный код, состоящий из двух подпоследовательностей двоичных цифр, одна из которых определяет код операции(сложить, умножить, переслать), вторая - определяет операнды, участвующие в операции и место хранения результата

Рассматриваемый процессор может работать с безадресными командами, одно-, двух-, и трехадресными командами. Команда в памяти может занимать от 1 до 15 байт и длина команлды зависит от кода операции, количества и места расположения операндов. Одноадресные команлы могут работать с операндами, расположенными в памяти и регистрах, для двухадресных команд существует много форматов, такие как:
\begin{itemize}
    \item R-R 
    \item M-M 
    \item R-M 
    \item M-R 
    \item R-D 
    \item M-D
\end{itemize}
где R - регистр, M - память, D - данные

Операнды могут находиться в регистрах, памяти и непосредственно в команде и размер операндов может быть - байт, слово или двойное слово. 
Исполняемый адрес операнда в общем случае может состоять из трех частей

Существуют различные способы адресации операндов, такие как:
\begin{enumerate}
    \item регистровая
    \item непосредственная
    \item прямая
    \item косве
\end{enumerate}