\section{26.09.2024 лекция 3}
\subsection{Примеры использования адресаций}
фото1
Примеры команд с различной адресацией
\begin{enumerate}
    \item Регистровая
    \item Непосредственная

    MOV AX, 25 ; 25 -> AX
    CONST EQU 34h
    \item Прямая
    
    Если известен адрес памяти, начиная с которого размещается операнд, то в команде можно непосредственно указать этот адрес.
    
    MOV AX, ES : 0001

    ES - регистр сегмента данных, 0001 - смещение внутри сегмента

    Содержимое двух байтов, начиная с адреса (ES) + 0001 пересылаются в AX 

    Прямая адресация может быть записана с помощью символического имени, которое предварительно поставлено в соответствие некоторому адресу памяти, с помощью специальной директивы определения памяти.

    Например: DB "--- слово

    \item Косвенно-регистровая
    Данный вид адресации отличается от регистровой адресации тем, что в регистре содержится не сам операнд, в адрес области памяти, в которой операнд содержится

    MOV AX, [SI]

    Могут использоваться регистры:

    SI, DI, BX, BP, EAX, EBX, ECX, EDX, EBP, ESI, EDL

    Не могут использоваться: AX, CX, DX, SP, ESP.

    \item По базе со смещением
    
    MOV AX,[BX] + 2
    = MOV AX,[BX + 2]
    = MOV AX,2[BX]
    MOV AX,[BX + 4]

    \item Прямая с индексированием

    MOV AX, MAS[SI]

    MAS "--- адрес в области памяти.
    С помощью этой адресации можно работать с одномерными массивами. Символическое имя определяет начало массива, а переход от одного элемента к другому осуществляется с помощью содержимого индексного регистровая

    \item По базе с индексированием
    
    MOV AX, Arr[BX][DI]

    Эта адресация используется для работы с двумерными массивами. Символически имя определяет начало массива, с помощью базового регистра осуществляется перехолд от одной строки матрицы к другой, а с помощью индексного регистра "--- переход от одного элемента к другому внутри строки.
\end{enumerate}

\subsection{Особенности использования команд пересылки}
\begin{enumerate}
    \item Нельзя пересылать информацию из одной области памяти в другую;
    \item Нельзя пересылать информацию из одного сегментного регистра в другой;
    
    PUSH DS 

    POP ES

    \item Нельзя пересылать непосредственно операнд в сегментный регистр, но если такая необходимость возникает, то нужно использовать в качестве промежуточного одни из регистров общего назначения.
    \item Нельзя изменять командой MOV содержимое регистра CS.
    \item Данные в памяти хранятся в <<Перевернутом>> виде, а в регистрах в <<нормальном>> виде, и команда пересылки учитывает это, например,
    \item Размер передаваемых данных определяется типом оперндов в команде
    \begin{minted}{asm}
        X DB ?
        Y DW ?
        MOV X, 0
        MOV Y, 0
        MOV AX, 0
        MOV[SI], 0
    \end{minted}
    В последнем случае необходимо использовать \dots
    \item Если тип обоих операндов в команде определяется, то эти типы должны соответствовать друг другу
    \begin{minted}{asm}
        MOV AH, 500
    \end{minted}
    К командам пересылки относят команду обмена значений операндов.
    \begin{minted}{asm}
        XCHG OP1, OP 2
    \end{minted}

    Для перестановки значений байтов внутри регистра используют \textbf{BSWOP}
    \begin{minted}{asm}
        (EAX) = 12345678h
        BSWOP EAX; (EAX) = 78563412h
    \end{minted}
\end{enumerate}

\subsection{К командам пересылки относят:}
Команды конвертирования
\begin{minted}{asm}
    CBW ; безадресная команда
    CWD ;
    CWE ;
    CDF ;
\end{minted}

Команды условной пересылки \textbf{CMOVxx}
\begin{minted}{asm}
    CMOV AL, BL ; если
\end{minted}
Загрузка адреса.
\begin{minted}{asm}
    LEA OP1, OP2 ;вычисляет адрес OP2 и пересылает первому операнду, который может быть только регистром.
    LEA BX, M[BX][DI]
\end{minted}

\subsection{Команды и директивы в Ассемблере}
Три этапа обработки программы 
\begin{enumerate}
    \item Получаем машинный код (исходных модулей может бвть один или несколько)
    \item Модули объединяются в один исполняемый модуль .exe(для .com нужно выполнить ещё один этап обработки .exe -> .com с помощью системной программы или в среде разработки с помощью ключа)
    \item Программы
\end{enumerate}
Программма состоит из команд, директив и комментарив.

Команды в процессе трансляции(асемблирования) преобразуются в машинный формат, директивы определяют способы ассемблирования и редактирования, выделяют место под исходные, промежуточные и окончательные данные/результаты, а комментарии используются для пояснения выполняемых действий.

Команда ассемблера состоит из 4 полей:

\begin{center}
    [<имя>[:]]<код операции>[<операнды>][;комментарии]
\end{center}

Директива, как и команда, состоит из 4 полей:

\begin{center}
    [<имя>[:]]<код псевдооперации>[<операнды>][;комментарии]
\end{center}
Здесь <имя> "--- символическое имя Ассемблера,

Здесь <код псевдооперации> "--- определяет назначение директивы.

Операндов может быть различное количество и для одной директивы.

Например:
\begin{minted}{asm}
    1
\end{minted}

Исходный модуль на Ассемблере "--- последовательность строки

\begin{minted}{asm}
    ; сегмент стека
        Sseg Segment...



    ;



    ;
\end{minted}

В кодовом сегменте специальная директива\dots
\begin{minted}{asm}
    ASSUME SS:SSeg, DS::DSeg, CS::CSeg, ES::DSeg;
\end{minted}

на DSeg отсылаются и DS, и ES

дальще по фоткам \textbf{Я УСТАЛЬ}



