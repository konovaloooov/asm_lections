\section{04.10.2024 лекция 5}
\subsection{Арифметические операции}
фото 1
Пример, работая в байтами, получим:
\begin{center}
    $250 + 10 = (250 + 10) mod 2^{s} = 260 mod 256 = 4$ 
\end{center}

Пример: в байте 
\begin{center}
    $1- 2 = 1 + 2^{8} - 2 = 257 - 2 = 255, CF = 1$ 
\end{center}

Сложение(Вычитание) знаковых чисел сводится к сложению(вычитанию) с использованием дополнительного кода.
\begin{center}
    $1$ 
\end{center}
В байте:
\begin{center}
    $-1 = 256 - 1 = 255 = 11111111$,
    $-3 = 256 - 3 = 253 11111101$,
    $3 + (-1) = (3 + (-1)) mod 256 = (3+ 255) mod 256 = 2$,
    $1 + (-3) = (1 + (-3)) mod 256 = 254 = 11111110$,
\end{center}
Ответ получили в дополнительном коде, следовательно резульатт получаем в байте по формуле $X = 10^{n} - |X|$, т.е.
x = 256 - 254 = |2| и знак минус. Ответ -2.

Переполнение происходит, если есть перенос из старшего цифрового в знаковый, а из знакового нет и наоборот, тогда OF = 1. программист сам ре (фото)

\subsection{Сложение и вычитание в Ассемблере}
Арифметические операции изменяют значение флажков OF, CF, SF, ZF, AF, PF

В Ассемблере команда "+":
\begin{minted}{asm}
    ADD OP1, OP2 ;(OP1) + (OP2) -> OP1
    ADC OP1, OP2 ;(OP1) + (OP2) + (CF) -> OP1
    XADD OP1, OP2 ;i486 и >
    (OP1) <-> (OP2) (меняет местами), (OP1) - (OP2) -> OP1                                

\end{minted}

В Ассемблере команда '-':


МНОГО ФОТОК!

последнее 
\subsection{Команды безусловной передачи управления}
Команда вызова процедуры:
\begin{center}
    \textbf{CALL <имя> ;}
\end{center}
Адресация может быть использована как прямая, так и косвенная.

При обращении к процедуре типа NEAR в стеке сохраняется адрес возврата, адрес команды, следующей за CALL содержится в IP или EIP.

При обращении к процедуре типа FAR в стеке сохраняется полный адрес возврата CS:EIP.

Возврат из процедуры реализуется с помощью команды \textbf{RET}.

Она может иметь один из следующих видов:
\begin{minted}{asm}
    RET [n]      ;возврат из процедуры типа NEAR и из процедуры типа FAR
    RETN [n]     ;возврат только из процедуры типа NEAR
    RETF [n]     ;возврат только из процедуры типа FAR
\end{minted}
Параметр n является необязательным, он определяет какое количество байтов удаляется из стека после возврата из процедуры.

фото


\subsection{Команды условной передачи управления}
Команды условной передачи управления можно разделить на 3 группы:

команды, используемые после команд сравнения
команды, используемые после команд, отличных от команд сравнения, но реагирующие на значения флагов
\begin{minted}{asm}
JZ/JNZ
JC/JNC

\end{minted}
