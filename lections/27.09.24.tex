\section{27.09.2024 лекция 4}
\subsection{Слова, константы, выражения, переменные}
фото 1

Строковые данные "--- последовательность символов, заключенная  в апострофы или кавычки

Также,как и в языках программирования высоко уровня, в Ассемблере могут использоваться именованные константы. Для этого существует специальная директива EQU.

Например,

M EQU 27 : директива EQU присваивает имени М значение 27.

Переменные в Ассемблере определются с помощью директив определения данных и памяти, например,

\begin{minted}{asm}
    v1 DB
\end{minted}

Арифметические операции: +, -, *, /, mod

Логические операции: NOT, AND, OR, XOR

Операции отношений: LT, LE, EQ, NE, GT, GE

Операции сдвига: сдвиг влево(SHL), сдвиг вправо(SHR)

Специальные операции: offset, PTR

Метка или переменная:

PTR
BYTE 


\subsection{Директива определения}

Общий вид директивы определения следующий

где X B, W, D, F, Q, T 

В поле операндов может быть "?". одна или несколько констант, разделенных запятой. Имя, если оно есть, определяет адрес первого байта выделяемой области. Директивой выделяется указанное количество байтов ОП и указанные операнды пересылаются в эти поля памяти. Если операнд "?", то в соответствующее поле ничего не заносится.

\begin{enumerate}
    \item Если операндом является символическое имя IM1, которое соответсвует смещению в фрагменте 03AC1h, то после выполнения 
    
    M DD IM1

    будет выделено 4 байта памяти. Адрес "--- М. Значение "--- 03АС1h.

    \item если необходимо выделить 100 байтов памяти и заполнить 1, то это можно сделать с помощью специального повторителя DUP
    
    D DB 100 DUP(1)

    \item Определение одномерного массива слов, адрес первого элемента массива "--- имя \textbf{MAS}, значение его 1.
    \item Определение двумерного массива:
    \begin{minted}{asm}
        v1 DB
    \end{minted}
    \item \textbf{const EQU 100}
    
    D DB Const DUP(?) ;выделить 100 байтов памяти. В директиве определения байта(слова) максимально допустимая константа 255(65535).

    С помощью директивы определения байта можно определить строковую константу длинной 255 символов, а с помощью определения 
\end{enumerate}

\subsection{Команда прерывания Int, команды работы со стеком}

С помощью этой команды приостанавливается работа процессора, управление передается DOC или BIOS и после выполнения какой-то системноц обрабатывающей программы, кправление передается команде6 следующей за командой INT.

Выполянемые действия будут зависеть от опернада, параметра команды INT и содержания некоторых регистров.

Например, чтобы вывести на экран "!" необходимо:
\begin{minted}{asm}
    v1 DB
\end{minted}
Сткек определяется с помощью регистров SS и SP(ESP).

Сегментный регистр SS содержит адрес начала сегмента стека.

ОС сама выбирает этот адрес и пересылает его в регистр SS.

Регистр SP указывает на вершину стека и при добавлении элемента стека содержимое этого регистра уменьшается на длину операнда. 

Добавить элемент в стек можно с помощью команды

где операндом может быть как регистр, так и переменная.

Удалить элемент с вершины стека можно с помощью операции


Для i386 и > PUSH A/ POP A позволяют положить в стек, удалить содержимое всех регистров общего назначения в последовательности AX, BX, CX, DX, SP, BP, SI, DI



Пример программы, использующей комнады пересылки содержимого 4 байтов из одной области памяти в другую и вывод на экран.
\begin{minted}{asm}
    TITLE Prim.asm
    Page, 120
    ;описание сегмента стека
    Sseg Segment Para stack 'stack'
        DB 100h DUP(?)
        Sseg ends
    ;описание данных
    Dseg Segment Para Public 'Data'
        DAN DB '1', '3', '5', '7'
        REZ DB 4 DUP (?)
    Dseg ends
    ;
    ;
    Cseg Segment Para Public 'Code'
\end{minted}

\subsection{Директива сегмента}
Общий вид

Любой из операторов может отсутствовать.
\begin{enumerate}
    \item Если есть readonly, то будет выведено сообщение об ошибке при попытке записи в сегмент.
    \item Операнд выравнивание определяет адрес начала сегмента.
    BYTE "---
    WORD "---
    DWORD "---
    Para "---
    Page "---
    \item тип определяет тип объединения сегментов.
    Значение stack указывается в сегменте стека, для остальных сегментов public. Если такой параметр присутствует, то все сегменты с одним именем и различными классами объединяются в один  последовательно в порядке их записи.
    Значение 'Common' говорит, что сегменты с одним именем объединены, но нне последовательно, а с одного и того же адреса так, что общий размер сегмента будет равен не сумме, а максимуму из них.
    Значение
\end{enumerate}

\subsection{Точечные директивы}
В программе на Ассемблере могут использоваться упрощенные(точечные) директивы.
.MODEL "--- директива, определяющая модель выделяемой памяти для программы.

Модель памяти опредлеяется параметром:

tiny "--- под всю программу выделяется 1 сегмент памяти.

small "--- под данные и под программу выделяются по одному сегменту.

medium "--- под данные выделяется один сегмент, под программу выделяется несколько сегментов.

compact "--- под программу выделяется один сегмент, под данные выделяется несколько сегментов.

large "--- под данные и под программу выделябтся по n сегментов.

huge "--- позволяет использовать сегментов больше, чем позволяет ОП. (можно на внешний)

\begin{minted}{asm}
    .MODEL small
    .STACK 100h
    .Data
    St1 DB 'Line1, $'
    St2 DB 'Line2, $'
    St3 DB 'Line3, $'
    .CODE 
    begin: MOV AX, @DATA
        MOV DS, AX
    MOV AH, 9
    MOV DX, offset St1
    INT 21h
    MOV DX, offset St2
    INT 21h
    MOV DX, offset St3
    INT 21h
    MOV AX, 4C00h
    INT 21h
    END begin
\end{minted}

'\$' "--- конец строки, которую необходимо вывести на экран

В результате выполнения программы:

Line1

Line2

Line3,



\subsection{Com-файлы}
После обработки компилятором и редактором связей получаем exe-файл, который содержит блок начально загрузки, размером не менее 512 байт, но существует возможность создания другого вида исполняемого файла, который может быть получен на основе exe-файла 