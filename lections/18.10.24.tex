\section{11.10.2024 лекция 7}
фото до 10:58
\subsection{К}
Имя первоц структуры dst, второй "--- dst+4, третьей "--- dst+8 и т.д.

Работать с полями структуры можно также, как с полями переменной комбинированного типа в языках высокого уровня:
\begin{center}
    <имя переменной> . <имя поля>
\end{center}
Например, dt1.y, dt2.m, dt3.d 

Ассемблер (дальше по фото 10:57)

Точка, указанная при обращении к полю, это оператор Ассемблера, который вычисляет адрес по формуле:
\begin{center}
    <адресное выражение> + <смещение поля в структуре>
\end{center}
Тип полученного адреса совпадает с типом поля, т.е.
\begin{center}
    type(dt1.m) = type m = byte
\end{center}
адресное выражение может быть любой сложности, например:
\begin{enumerate}
    \item mov AX, (dts+8).y
    \item mov SI,
    inc (dts[SI]).m ; Аисп = (dts + (SI)).m = (dts + 8).m
    \item lea BX, dt1
    mov[BX].d, 10   ; Аисп = (BX) + d = dt1.d
\end{enumerate}

Замечания:
\begin{enumerate}
    \item  1
    \item 
\end{enumerate}

Одно исключение: если поле описано как строка, то оно можеит иметь начальным значением строу той же длины или меньшей, в последнем случае строка дополняется справа пробелами.
Например:
\begin{minted}{asm}
    student struc
            f DB 10DUP(?)   ;
            i DB
\end{minted}
Примеры программ с использванием данных типа структура.
\begin{minted}{asm}
    ; prim.asm - прямое обращение к полям структуры
        .model tiny
        .code
        org 100h    ;
    Start: mov AH, 9
        mov DX, offset message
        int 21h

        lea DX, st1.s 
        int 21h
        lea DX, st1.f
        int 21h
        lea DX, st1.i 
            int 21h
            ret
        message DB "hello", 0dh, 0ah, "$"

        tst struc   ;
                    s DB "student", "$"
                    f DB "Ivanov", "$"
                    i DB "Ivan", "$"
                tst ends
                st1 tst < > ;описание переменной типа tst
            end start

\end{minted}

\begin{minted}{asm}
                ; prim.asm - обращение к полям структуры в цикле
    Start: mov AH, 9
            mov DX, offset message
            int 21h
        mov SI, 0
        mov CX, 3
    m1: lea DX, st1[SI]
        int 21h
        add SI, 9
        loop m1
        ret
    message DB "hello", 0dh, 0ah, "$"
        tst struc   ;описание типа структуры
                    s DB "student", "$"
                    f DB "Ivanov", "$"
                    i DB "Ivan", "$"
                tst ends
                    st1 tst < > 
                        end start


\end{minted}